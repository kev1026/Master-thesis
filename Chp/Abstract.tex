\begin{abstract}
  非结构网格计算方法是进行复杂区域流场模拟的有效手段. 随着应用中的
  计算问题越来越复杂, 非结构网格处理复杂边界的灵活性以及良好的自适
  应性优势逐渐凸显出来. 但是非结构网格几何上的随机性, 给流场算法的
  设计带来很大的困难. 如何构造非结构网格的高精度高分辨率并且紧致的
  计算格式是当前非结构网格有限体积法和间断有限元法研究中的主要问
  题. 本文尝试对现有的非结构网格有限体积法 MUSCL 类型重构策略进行
  修改, 并将其推广应用到间断有限元方法中. 主要完成了以下工作,
  \begin{itemize}
  \item[(1)] 发展了一种针对 MLP (Multi-dimensional Limiting
    Process)限制器的实施策略. MLP 限制器的优点是在抑制数值震荡的同
    时只引入很小的耗散. 但是由于限制过程需要涉及到所有与本单元共享
    顶点的相邻单元, 散失了MUSCL 重构策略那样的紧致性. 为避免在限制
    本单元时访问过多的邻居单元, 我们提出了一种优化的实现策略. 预先
    将最大最小值信息归集到与顶点相对应的数据结构中. 在计算过程中只
    需访问本单元顶点就可以获取所需的邻居单元信息, 从而避免对过多邻
    居单元的直接访问.
  \item[(2)] 在二维非结构三角形网格有限体积/间断有限元法程序中实现
    了 MLP 限制器及上述优化的实现策略. 通过典型的数值
    算例 (线性对流问题、旋转对流问题), 对程序进行了验证. 对比研
    究表明 MLP 限制器比传统的 MUSCL 重构策略可获得更好的数值结果.
  \item[(3)] 对原由规则网格剖分非结构四面体的并行自适应有限体积法
    程序(libfvphg 库)进行了改进. 利用开源网格生成工具 GMSH 为程
    序提供了完整的非结构四面体网格输入接口. 初步实现了 MUSCL 重构
    策略。在最多达 1024 个进程的规模上对程序的并行可扩展性进行了测
    试, 比较了在几个问题中的并行效率.
  \end{itemize}

  {\bf{关键词}}: 双曲守恒律, 有限体积法, 间断有限元法, MUSCL 重构,
  MLP 限制器
\end{abstract}

\begin{englishabstract}
  {\todo}\\
  \englishkeywords{hyperbolic conservation law, {\todo}}
\end{englishabstract}

% 主要说明本论文的研究目的 、 内容 、方法 、 成果和结论 。 要突出本
% 论文的创造性成果或新见解,不宜使用公式 、 图表,不标注引用文献 。 英
% 文摘要( Abstract)应与中文摘要内容相对应 。

