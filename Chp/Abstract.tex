\begin{abstract}
  非结构网格计算方法是进行复杂流场区域模拟的有效手段. 随着计算工况
  越来越复杂, 非结构网格方法处理复杂边界的灵活性以及良好的自适应特
  性逐渐凸显出来. 但是非结构网格单元之间几何上的随机性, 给算法的设
  计带来很大的困难. 如何设计应用于非结构网格的高精度高分辨率并且足
  够紧致的计算格式是当前的前沿课题. 其中有限体积方法以及高阶间断有
  限元方法的相关研究是该领域的主要研究方向. 本文尝试对非结构网格有
  限体积法 MUSCL 类型重构策略进行分析和改进, 并将其应用到间断有限
  元方法中. 主要完成了以下工作.

  \begin{itemize}
  \item[(1)] 发展了一种针对 MLP 限制器的实施策略.
    % MLP 限制器引入了 一种较为宽松的 MP 稳定性条件, 用所有与当前单元共享节点的单元做
    % 为提供上下限的模板.
    数值实验说明 MLP 限制器在抑制震荡的同时只引入很小的耗散. 但是
    由于限制过程需要涉及到所有与本单元共享节点的单元, 破坏
    了 MUSCL 重构策略的紧致性. 为避免在限制本单元时访问过多的邻居
    单元, 我们提出了一种优化的实现策略. 将最大最小值信息归集到与节
    点相对应的数据结构中. 实现了以单元顶点作为媒介对邻居单元信息的
    进行获取, 从而避免在计算过程中对过多邻居单元的直接访问.
  \item[(2)] 在二维非结构网格间断有限元法程序中实现了 MLP 限制器及
    其优化的实施策略. 通过典型的数值算例, 如线性对流问题、旋转对流
    问题等, 对程序进行了验证. 并与传统的 MUSCL 方法进行了对比, 取
    得了很好的数值结果.
  \item[(3)] 对三维非结构并行自适应有限体积法 libfvphg 库进行了改
    进. 基于网格生成工具 GMSH 为程序提供了完整的非结构网格输入接
    口. 实现了 MUSCL 重构策略, 使得解法器可以处理不规则的四面体非
    结构网格. 在最多达 1024 个进程的规模上对程序的并行可扩展性进行
    了测试, 比较了多种问题背景下的并行效率.
  \end{itemize}

  {\bf{关键词}}: 双曲守恒律, 有限体积法, 间断有限元法, MUSCL 重构,
  MLP 限制器
\end{abstract}

\begin{englishabstract}
  {\todo}\\
  \englishkeywords{hyperbolic conservation law, {\todo}}
\end{englishabstract}

% 主要说明本论文的研究目的 、 内容 、方法 、 成果和结论 。 要突出本
% 论文的创造性成果或新见解,不宜使用公式 、 图表,不标注引用文献 。 英
% 文摘要( Abstract)应与中文摘要内容相对应 。
