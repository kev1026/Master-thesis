
\chapter{结论与展望}
\label{chap:summ-expec}

\section{结论}
\label{sec:jielun}

% 结论
本文主要讨论了针对双曲守恒律的非结构网格计算方法. 重点关注高维非结
构网格 MUSCL 类型的重构策略. 提出了一种 MLP 斜率限制器的实现方案,
并将其应用到间断有限元程序中. 完善了 libfvphg 库, 并进行相关测
试. 具体的工作内容如下,

% 1. 关于 MLP 限制器的改进
% 2. 关于自己编写的二维间断有限元程序.
% 3. 关于在 libfvphg 方面的工作.

\begin{enumerate}
\item 针对 MLP 限制器模板包含所共享顶点的单元这一特点, 给出了一种
  优化的实现策略. 该实现策略首先将单元信息归集到依附于节点的数据结
  构中. 在对当前单元进行限制器操作时, 只需要访问该单元顶点处所归积
  的信息. 从而避免了对邻居单元的直接访问, 在很大程度上克服了MLP 限
  制器由于模板过宽而对其实施的制约.
\item 开发了二维非结构网格间断有限元程序. 将有限体积法 MUSCL 重构
  策略中的限制器部分应用于程序中, 用于抑制数值震荡. 实现了优化
  的 MLP 斜率限制器, 并与传统的斜率限制器进行了对比. 在数值算例方
  面, 用线性对流问题检验了格式的收敛阶数. 而方形波问题和旋转对流问
  题则分别用于检验格式对虚假震荡的抑制能力以及对剧烈变化区域的解析
 程度. 数值结果说明 MLP 限制器是一种真正具有多维性质的限制策略,
  可以很好的刻画高维流场的多维性质.
\item 完善了三维非结构网格有限体积法 libfvphg 库. 基于 GMSH为其提
  供完整的非结构网格输入接口. 引入 MUSCL 类型重构策略, 使得程序可
  以应用真正的三维非结构计算网格. 使用三维激波管问题和三维爆炸问题
  作为算例检验了程序的正确性, 并对其强弱可扩展性进行了测试.
\end{enumerate}

\section{展望}
\label{sec:zhanwang}
% 下一步的工作
% 1. 将基于节点进行信息传递的思想推广到高阶方法.
% 2. 在边界处的处理
% 3. 误差指示子的改进?
由于时间限制, 本文的工作仍有许多需要完善的地方. 另外, 基于当前的
工作也有许多值得深入研究的问题. 主要有以下几个方面,
\begin{enumerate}
\item 将所提出的 MLP 斜率限制器的实现策略推广到高阶的情形. 该实现
  策略的核心思想是以节点作为媒介, 将所邻居单元的信息传递到当前计算
  单元. 在当前的工作中, 传递的信息只是单元的平均值信息. 注意到间断
  有限元方法还包含除单元平均值之外的高阶自由度. 因此如何整合这些高
  阶信息并以同样的方式传递给计算单元是一个很值得深入研究的问题.
\item 高阶间断有限元方法的计算效果对边界处理精度有很高的敏感性. 如
  何有效的处理曲边界, 以及如何将 MLP 限制器应用到这种复杂边界情形
  是需要考虑的问题.
\item 尽管在已经做好的测试算例中 Libfvphg 已经可以正确的给出结
  果. 但是目前的计算结果来看, 其自适应加密模块仍然十分粗糙. 寻找有
  效的后验误差指示子以及自适应策略仍是今后工作的一个重要方面.
\end{enumerate}
