
\chapter{引言}
\label{chap:introduction}

计算流体力学(CFD, Computational Fluid Dynamics)蓬勃兴起
于上世纪六七十年代, 主要受航空航天领域的需求所驱动.  但是由于当时计
算机运算速度和存储能力不足, 计算对象主要限制在二维流动.
到1980年代中期, 一些研究团队相继实现了三维流体模型的数值模
拟\cite{Venkatakrishnan1996}. 如今随着计算机性能的大幅提升和半个多
世纪以来相关计算方法的巨大进步, CFD 已经发展到可作为计算机辅助工程 (CAE, Computer
Aided Engineering) 的一个重要分支的程度,  成为科学研究和制造
业不可或缺的设计和分析工具, 广泛应用于航空航天、汽车工业、土木工程、
石油工业等应用领域 \cite{anderson1995book, blazek2006}.

\section{双曲守恒律的数值求解}
\label{sec:hyperbolic-conservation-law}

双曲型守恒律在航空航天、天体物理、地球物理、计算生物学及交通
模拟等领域\cite{Dumbser2016,Colombo2002}中都有着广泛的应用.
双曲守恒律的数值求解在计算数学、CFD等学科领域中占着重要位置.
其重要性主要体现在下面几个方面 \cite{Toro2009} :
\begin{itemize}
\item 描述无粘可压缩流动的欧拉方程是双曲方程. 而欧拉方程是空气动力
  学最基本的模型之一.
\item \sout{对于广义的双曲型偏微分方程, 如何处理方程中的双曲型部分往往
    是算法设计的最大挑战.}
  % \cite{LeVeque1992a}
  {\color{red} 不懂? 很大一部分原因来自于双曲方程缺少
    耗散项而容易出现数值耗散甚至不收敛的情况. }\\
  对于一般形式的流体力学控制方程, 如何处理方程中的双曲型部分往往是
  算法设计的最大挑战 \cite{Toro2009}.\\
  ({\color{blue} 第一句来源于 Toro 教材 Chp.2 的引言部分, 原文:
    (ii) Numerically, it is generally accepted that the
    hyperbolic terms of the PDEs of fluid flow are the terms that
    pose the most stringent requirements on the discretisation
    techniques.\\
    后面一句是从其他地方读到的观点 (未找到出处), 大意是说 Euler 方
    程因为缺少像 N-S 方程的耗散机制, 所以数值求解过程更容易发散.
    如果不合适考虑只保留第一句话.})
\item 双曲守恒律有着相对丰富的数学理论成果支
  持 \cite{DiPerna1979,Lax1973,Glimm1965}, 这为数值方法的设计提供
  了稳定性收敛性等方面的理论依据. 尽管只有在极其特殊情况才有精
  确解, 但是这些特殊算例为数值方法的检验提供了重要的依据.
\end{itemize}
% 欧拉方程的计算是双曲守恒律计算的基础问题之一. 常常做为检验算法优劣
% 的基础模型. 许多应用于双曲守恒律的计算方法最早是针对欧拉方程发展起
% 来的. 在这篇文章中我们只处理对流方程和欧拉方程两种基础模型.

一般情形下的双曲守恒律会因为其非线性而产生间断解, 这会给数值求解造成
困难. 研究表明,通常希望数值格式需要同时满足下面几个要
求 \cite{LeVeque1992a,Harten1983}:
\begin{itemize}
\item 在间断之外的光滑区域至少有二阶的数值精度.
\item 在间断处具有高分辨率. 这意味着数值解在间断处与光滑区域之间
  的过渡网格要足够少.
\item 为保持计算稳定性, 能够避免数值震荡的产生.
\item 为了保证数值解在网格步长趋于零的情况下可以渐进的收敛到真实的物理
  解, 格式要满足相容性、稳定性以及某种形式的熵条件.
\end{itemize}
满足这些条件的数值格式通常称为高分辨率格式. 自从 Harten
\cite{Harten1983} 1983 年提出该概念以来便成为判断算法优劣的一个重要标准.

绝大多数实际问题需要在二维和三维的物理空间中求解双曲守恒
律. 而如何将一些好的成熟的一维方法推广到高维情形, 也给方法的发展带来很多的挑战, 尤
其是推广到非结构网格以应对实际应用问题出现的复杂的多维计算区域.


\section{高分辨率格式及其在非结构网格上的推广}
\label{sec:unstru-cfd}

同时达到高阶精度和无数值震荡是双曲守恒律计算研究中长期追求的目标。
Godunov 定理说明,  理论上数值精度高于一阶的线性格式在间断附近无法避免数值震荡的
产生 \cite{Godunov1959,engquist1981}.
% 早期的 TVD 条件.
为了克服 Godunov 定理带来的限制, 必须引入非线性格式. 所谓非线性格
式指的是即便应用于线性问题, 格式也是非线性的. 非线性格式早期主要有两
种设计思路. 第一种方法是所谓添加人工粘性. 其原理是在间断附近添
加足够的人工粘性来抑制或减弱数值震荡, 而在远离间断的地方保持原始数
值精度. 这种方法的缺点在于人工粘性的构造和具体参数的设置是依赖于问
题本身的, 因此通用性较差. 第二种方法是使用限制器来抑制数值震荡. 这
种方法的本质是在间断附近降低格式的精度阶以达到抑制数值震荡的
效果, 而在光滑区域保持原始精度. 代表性的方法有通量限制器和斜
率限制器\cite{HARTEN1972568,Harten1983,Sweby1984,Leer1979}.

在抑制数值震荡的理论方面, Godunov \cite{Godunov1959} 首先引入单调
性的概念, 认为满足某种单调性是抑制震荡的必要条件, 并以此为准则发展
出所谓单调格式. 但是单调格式有较大的数值耗散, 至多只有一阶
精度. van Leer在他的系列文
章\cite{Leer1973,VANLEER1974361,VanLeer1977263,Leer1977,Leer1979}
中提出的 MUSCL 格式是克服 Godunov 定理对精度阶数限制的最早尝试之一.
Harten (1983)\cite{Harten1983} 引入总变差不增(Total Variation
Diminaishing, TVD) 的概念. TVD 条件的提出是双曲守恒律计算重要的进
步, 成为之后双曲守恒律计算格式的设计的主要理论依据之一. 上文中提到
的通量限制器和斜率限制器是两种设计 TVD 格式的手段. 为了克服 TVD 方
法在极值点处的降阶问题, Shu \cite{Shu1987} 引入 TVB 条件. 此后
以 Harten 为代表的学者\cite{Harten1987231}提出本质无震荡 (Essentially Non-Oscillatory, ENO) 的思想, 发
展了本质无震荡格式, 通过自适应选择计算模板的方式将二
阶 MUSCL 重构推广到任意高阶, 从另外一个角度克服了TVD 格式在极值点
降阶的问题. 在其基础上 Osher, Shu 等学者\cite{Liu1994200,Jiang1996}进一步发展了加权本质无震
荡(Weighted Essentially
Non-Oscillatory, WENO)格式. WENO 格式是目前高阶高分辨率格式的主流方法之
一 \cite{Wang2007}.

% =================================================================

上文提到的绝大多数计算格式都是针对一维情形.  而实际应用中的计算问
题大多是二维三维问题. 因此多维双曲守恒律计算的一个挑战就是如何将相
对成熟的一维高分辨率计算方法推广到多维. 这对于结构网格和有限差分法
来说相对容易,但对于非结构网格限体积法就比较麻烦 \cite{Deng2012}.

随着 CFD 计算技术的发展, 非结构网格技术逐渐受到更多的关注. 实际应
用问题, 例如全机模型的模拟、发动机模拟等都涉及到复杂的边界外形. 生
成复杂区域的结构网格是很困难的. 结构网格的代表性方法如嵌套网
格 (overset grid, chimera grid) 和多块网格(multi-block) 技术需要大
量的人工操作\cite{Venkatakrishnan1996,overset1983,Vatsa1993}. 而非
结构网格生成技术, 如常用的 Delaunay 方法和阵面推进法 (Advancing
Front Method)\cite{Anderson199423,Blazek2001,Ldhner1988}, 则可以方
便的实现自动网格生成, 从而在很大程度上降低网格生成的时间开销和人工
操作. 另外, 非结构网格也具有结构网格所不具备的灵活性. 在非结构网格
上可以更方便的应用自适应算法. 但非结构网格算法的缺点是计算量和内存
需求大,计算效率以及边界层的计算精度都较低。

{(\color{red}注意区分单调和保单调的概念})有效的保单调条件在非结构
网格高分辨率有限体积法的发展中发挥了重要作用. Spekreijse
\cite{spekreijse1987} 最早引入高维单调性概念, 并以此为依据设计了一
系列高维保单调格式. Barth \cite{Barth1989}, Liu
\cite{Liu1993} 和 Batten \cite{Batten1996} 等在此基础上发展出一系
列针对高维非结构网格双曲守恒律的斜率限制策略.
Jameson\cite{Jameson1995} 提出的 Local Extremum Diminishing (LED)
准则是对一维 TVD 准则在高维情形中的一个推广.  Hubbard (1999)
\cite{Hubbard1999}注意到高维格式的设计大多要遵循某种形式的局部极大
值准则 (Local Maximum Principle), 提出斜率的极大值准则区域 (MP
region)的概念, 为相关的算法设计提供了统一的框架. 这使得斜率限制器
的设计可以在更大程度上体现流动的高维性质. Hoteit
\cite{Hoteit2004}和 Buffard \cite{Buffard2010}提出在 MP region 上
求解一个约束优化问题从而得到最合理的斜率估计. 但是求解优化问题的过
程较为复杂.  Park和Kim等\cite{Park2010,Park2011,Park2014}通过扩
大 MP 条件模板的方式引入 MLP (Multi-dimensional Limiting Process)
条件, 实现对多维性质更精细的刻画.

文献 \cite{Wang2007,Wang2013} 总结了最近几年CFD在非结构网格高精度方法
方面的研究进展. 尽管高精度方法可以刻画更精细的解结构, 但是在健壮性、内存占用、限制器等方面仍然处于不完
善的状态, 算法的设计仍十分困难. 所以目前工业
界仍然以相对成熟的二阶 MUSCL 类型的格式为主流计算方法. 近年来有一
系列对 MUSCL 类型方法的改进工作. Clain等\cite{Clain2010}和Buffard\cite{Clain2010,Buffard2010}等提出了多斜率
的 (multislope) MUSCL 方法, 在每一个单元界面方向上引入一个单独斜
率, 使得斜率的求解可以退化到一维的情形. 但是网格需要满足一定的正
则性以确保格式的稳定性. Hou (2014)\cite{Hou2015} 将多斜率的 MUSCL方
法成功应用于求解二维浅水方程. Touze (2015)\cite{LeTouze2015} 则将这种方法推广到更一般的非结构网格, 并且消除了对
网格正则性的要求. 这种方法的有效性还需要进一步检验。我们认为上文提到的 MLP 限制策略是对传统 MUSCL 型方法的一个重要改进. MLP 限制策略在传统的单斜率的 MUSCL 型方法的基础上, 通过扩大限
制器模板的方法减少数值耗散, 以达到更高的计算精度. 但是限制过程需要
涉及到所有与本单元共享节点的单元.  因此这种模板相比只用面相邻单元 (von
Neumann邻居单元) 作为限制器模板要宽很多, 这一问题在三维的情形尤其突
出. 为避免在用MLP限制本单元时访问过多的邻居单元, 本文设计了一种新的计算
策略, 避免了对邻居单元的直接访问.

\section{非结构网格上的间断有限元方法({\color{red}只写和文本有关的方法回顾)} }
\label{sec:dg-method}

间断有限元方法 (Discontinuous Galerkin Method, DGM) 最早是 1973 年
由 Reed、Hill \cite{Reed1973} 提出, 并应用于中子输运方程的求解.
Shu和Cockburn \cite{Cockburn1989,Cockburn1990,Cockburn1998}等将其
推广到求解双曲守恒律问题. 该方法结合了有限元和有限体积法各自的优点.
一方面,类似于有限元法, 使用高阶多项式逼近而不是更宽的模板来达到高
阶精度.  另一方面, 单元边界处因为解的间断性而需要求
解 Riemann 问题, 采用有限体法中相同的处理方式. 事实上一阶的间断有
限元 $\text{DG}(0)$ 就等同于传统的 cell-centered 有限体积法.  从这
个意义上来说, 高阶的间断有限元 $\text{DG}(p), p>0$ 可以看做是有限
体积法的自然推广. 间断有限元方法主要有下面一些优点:
\begin{itemize}
\item 作为有限元方法的一种特殊情况, 间断有限元法可以十分自然的应用
  到非结构网格上, 甚至可以适用于更一般的网格, 例如任意多面体或者带
  有``悬点''的非协调网格.
\item 每个计算单元相对独立, 只与面邻居单元有关, 计算过程具有紧致
  性. 这使得间断有限元可以达到极高的并行效
  率\cite{Juan2009,Remacle2003,Biswas1994}.
\item 基函数构造的灵活性使得 hp-自适应策略可以方便的应用于间断有限
  元方法.
\item 有良好的数学理论作为支撑, 甚至可以证明其满足某些形式下的熵条件\cite{Jiang1994,Hou2006}.
\end{itemize}

然而, 间断有限元法同样面临在解的间断处产生数值震荡的问题. 需要在每
一个时间步后应用一定的限制策略来抑制震荡. 有限体积法中的部分限制器
可以直接应用到间断有限元法, 例如上文提到的基于斜率限制器的限制策
略. 这种限制策略可以自然地适用于一阶间断有限元$\text{DG}(1)$. 进行
限制时为了达到某种稳定性条件, 只需要修改代表斜率的一阶自由度. 但是
对于高阶间断有限元法例如模态基型 $\text{DG}(p), p>1$,除代表斜率的
一阶自由度之外, 还有更高阶的自由度.  一个常用的做法是引入一个判断
是否需要进行限制的判别准则, 所谓坏单元指示子 (Trouble Cell
Indicator). 仅在强间断附近等需要进行限制的单元上应用某种限制策
略,光滑区域仍保持原自由度不变.  文献 \cite{Qiu2005b} 对几种指示子
进行了总结和对比. 关于坏单元的处理, Shu和Cockburn
\cite{Cockburn1998}将需要进行限制的单元解投影到 $P_{1}$ 空间, 并应
用斜率限制器. 但是这样做将严重降低间断附近的精度.  同时, 坏单元指
示子通常难以区分极值点和强间断, 而在极值点处应用低阶限制器会破坏整
体解的高精度 \cite{Luo2007}. 为克服这一点, 近几年, 有一系列文章讨
论将 WENO与间断有限元方法相结合 \cite{Zhu2008,Zhu2013,Qiu2005a},
得到了精度更好的计算结果.  在这类方法中, 如果判断某个网格上需要对
数值解进行限制时, 则利用其周围几层网格上解的自由度重构一个与 DG 方
法同阶的 WENO多项式取代原来的解. 但是引入 WENO 的计算模板也在一定
程度上破坏了 DG 方法的紧致性. {\color{red}也有只用面邻居网
  格HWENO限制器(引用Qiu等,Hong Luo, Yidong Xia, Seth Spiegel,
  Robert Nourgaliev, Zonglin Jiang)}. {\color{blue} Zhu
  \cite{Zhu2009} 、Luo \cite{Luo2007,Luo2013} 等则引入 H-WENO 限制
  器, 在利用直接邻居单元元的单元平均值之外也使用其高阶自由度信息,
  从而在更紧致的模板上实现高阶重构.}

在本论文中, 我们将有限体积法的 MUSCL 类型斜率限制器和MLP限制器应用
于二维三角形网格P(1)间断有限元法,并将 MLP 斜率限制器的实现策略做了
一些改进, 取得了较好的计算效果.

\section{并行计算}
\label{sec:parallel}

近半世纪以来, 微处理器的性能提升主要依赖于制程尺寸的不断缩小, 即通
过增加单位面积上的晶体管数量来提高计算速度. 但是随着制程尺寸逐渐接
近硅材料的物理极限, 新工艺的研发难度剧增. 另一方面, 芯片的散热问题
也正逐渐接近空气制冷的极限 \cite{Pacheco2011}. 因此单纯的通过增加
微处理的性能来提高计算速度已经变得非常困难, 延续半个世纪之久的摩尔
定律正在逐渐失效 \cite{Moore1998,Waldrop2016}. 但是在现代航空航天、
医疗、金融、科研等众多领域中, \sout{对计算性能的需求仍是一个巨大
  的.} 有限的计算性能仍然是其发展的重要瓶颈. 尽管单个芯片的性能提
升正在放缓, 但我们可以通过并行计算的方式来继续提升计算性能. 并行计
算指的是基于某种通讯手段协同多个微处理器处理某个计算问题, 从而提高
计算性能. 这也几乎成为目前解决大规模科学计算问题的唯一有效途径.

并行计算在CFD中广泛的使用. 其处理方式通常是将计算区域分解为若干个
子区域. 每个进程负责处理其中的一块子区域,计算过程中通过处理器间的
数据通信实现整体的协同.

并行效率是并行计算的一个重要的衡量指标. 并行效率一方面受硬件环境的
影响, 例如通信的带宽以及芯片之间的性能差异. 另一方面并行效率在很大
程度上取决于计算方法. 如何提高并行效率是相关算法研究的一个重要课
题. 如果算法本身需要在进程之间进行大量的数据交换, 那么通讯时间可能
会占用计算时间的大部分. 因此, 子问题之间的关联程度越小的计算方法越
适合做并行化. 间断有限元法在计算过程只需要面相邻单元(von Neumann
邻居单元)的信息, 而有传统方法如限体积法则需要多层邻居单元的信息以
实现高阶精度. 这使得间断有限元方法在并行效率方面有着巨大的优势.

MPI (Massage Passing Interface) 分布式并行模式是主流的并行计算方式
和编程语言. 本文中, 我们在课题组前期编制的一个非结构四面体有限体积
法程序 (libfvphg库) 的基础上, 改进其代码瑕疵, 并实施前面提到
的 MLP 限制器的新策略. libfvphg 库利用了本所的 PHG 平台实现对非结
构四面体网格的并行计算。


\section{本文的主要内容}
\label{sec:main-containt}

本文的主要内容是关于守恒律的高维非结构网格解法器的实现以及 MUSCL
型斜率限制器的改进. 主要有以下三个方面:
\begin{itemize}
\item 在 MLP 斜率限制器的框架下, 提出了一种新的实现策略. 该策略
  利用单元结点作为相邻单元之间信息传递的媒介. 限制器作用过程中首先将数
  值信息归集到依托于结点的数据结构中, 在计算斜率限制器时, 只需访问
  结点处的数据, 因而避免了对模板内邻居单元的直接访问.
\item 编写了二维非结构网格间断有限元解法器. \sout{实施了 MUSCL型斜
    率限制器和以上策略的MLP限制器} 实施了以上 MLP 限制器的实现策略.
  ({\color{blue} MLP 重构可以归类于 MUSCL 类型重构}) 并通过数值实
  验, 比较了几种限制器的模拟结果.
\item 改进了基于 PHG 平台 \cite{Zhang2009} 的三维四面体网格并行自
  适应有限体积求解器 Libfvphg. 通过不同数值算例, 测试了并行效率.
\end{itemize}

本文的其余部分结构安排如下. 第二章介绍双曲守恒律有限体积法的空间离
散、数值通量、限制器、非结构网格生成等技术, 并简单介绍几种非结
构网格 MUSCL 类型斜率限制器方法及相关的理论. 第三章讨论间断有限元
法和 MLP 类型限制器在方法中的应用, 以及数值实验. 第四部分是关于对基于 PHG 的平台的有限体积
法求解器Libfvphg 的改进. 并通过几个三维典型算例, 进行模拟结果和并行效率测试. 最
后一章对论文工作进行总结和展望.
