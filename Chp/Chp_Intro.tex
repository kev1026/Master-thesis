
\chapter{引言}
\label{chap:introduction}

现代意义上的计算流体力学(CFD, Computational Fluid Dynamic)最早诞生
于上世纪六七十年代.  其发展由航空航天领域的需求所驱动.  但是由于计
算机运算速度和存储能力不足, 当时的计算对象主要限制在二维流动.  直
到1980年代中期, 才出现一些团队可以实现三维流体模型的模
拟\cite{Venkatakrishnan1996}.  如今随着计算机性能的大幅提升和半个
世纪以来计算方法领域的突破, CFD 作为计算机辅助工程(CAE, Computer
Aided Engineering) 的一个重要分支, 已经成为现代科学研究和高端制造
业不可或缺的设计和分析工具, 广泛应用于航空航天、汽车工业、土木工程、
石油工业等领域 \cite{anderson1995book, blazek2006}.

\section{数值求解双曲守恒律}
\label{sec:hyperbolic-conservation-law}

双曲守恒律的求解在计算数学、计算流体力学等学科中有着重要意义.
其重要性主要体现在下面几个方面 \cite{Toro2009} :
\begin{itemize}
\item 描述无粘可压缩流体的欧拉方程是双曲方程. 而欧拉方程是空气动力
  学最基本的模型之一.
\item 对于一般形式的 PDE, 如何处理方程中的双曲型部分往往是算法设计
  的最大挑战\cite{LeVeque1992a}. 很大一部分原因来自于双曲方程缺少
  耗散项而容易出现数值耗散甚至不收敛的情况.
\item 双曲守恒律有着相对丰富的数学理论支
  持 \cite{DiPerna1979,Lax1973,Glimm1965}, 这为数值方法的设计提供
  了稳定性收敛性等方面的理论依据. 尽管只有在极其特殊情况才有已知的精
  确解, 但是这些特殊算例为数值方法的检验提供了重要的依据.
\end{itemize}
% 欧拉方程的计算是双曲守恒律计算的基础问题之一. 常常做为检验算法优劣
% 的基础模型. 许多应用于双曲守恒律的计算方法最早是针对欧拉方程发展起
% 来的. 在这篇文章中我们只处理对流方程和欧拉方程两种基础模型.
另外, 双曲型守恒律在航空航天、天体物理、地球物理、计算生物学及交通
模拟等领域 \cite{Dumbser2016,Colombo2002} 都有着广泛的应用.

一般情形的双曲守恒律会因为其非线性而产生间断解, 这会给数值求解造成
困难. 通常希望数值格式需要同时满足下面几个要
求 \cite{LeVeque1992a,Harten1983},
\begin{itemize}
\item 在间断之外的光滑区域至少有二阶的数值精度.
\item 解在间断处具有高分辨率. 这意味着数值解在间断处与光滑区域之间
  的过渡单元要足够少.
\item 为保持计算稳定性, 能够避免数值震荡的产生.
\item 为了保障数值解在网格加密的情况下可以渐进的收敛到真实的物理
  解, 数值格式要满足相容性稳定性以及某种形式的熵条件.
\end{itemize}
满足这些条件的数值格式通常称为高分辨率格式. 自从 Harten 1983
\cite{Harten1983} 提出该概念以来便成为判断算法优劣的一个重要标准.

另外, 绝大多数实际问题需要在二维和三维的物理空间中求解双曲守恒
律. 而如何将成熟的一维算法推广到高维情形, 也给算法的设计带来很多大的挑战, 尤
其是推广到非结构网格以应对复杂的高维计算区域.


\section{实现非结构网格高分辨率计算}
\label{sec:unstru-cfd}

达到高阶精度的同时有效的控制虚假数值震荡是双曲守恒律计算领域的基础
课题之一. Godunov 定理说明, 低数值震荡和高数值精度是相互矛盾的两个
要求. 理论上数值精度高于一阶的线性格式在间断附近无法避免数值震荡的
产生 \cite{Godunov1959,engquist1981}.
% 早期的 TVD 条件.
为了克服 Godunov 定理带来的限制, 必须引入非线性格式. 所谓非线性格
式指的是即便应用于线性问题, 格式也是非线性的. 非线性格式早期主要有
有两种设计思路. 第一种方法是所谓添加人工粘性. 其原理是在间断附近添
加足够的人工粘性来抑制或减弱数值震荡, 而在远离间断的地方保持原始数
值精度. 这种方法的缺点在于人工粘性的构造和具体参数的设置是依赖于问
题本身的, 因此通用性较差. 第二类方法是使用限制器来抑制数值震荡. 这
种方法的本质在于, 在间断附近降低格式的精度阶数以达到抑制数值震荡的
效果, 而在光滑区域保持原始精度. 其中有代表性的方法为通量限制器和斜
率限制器\cite{HARTEN1972568,Harten1983,Sweby1984,Leer1979}.

在抑制数值震荡的理论方面, Godunov \cite{Godunov1959} 首先引入单调
性的概念, 认为满足某种单调性是抑制震荡的必要条件, 并以此为准则发展
出所谓保单调格式. 但是保单调格式引入了较大的数值耗散, 至多只有一阶
精度. van Leer在他的系列文
章\cite{Leer1973,VANLEER1974361,VanLeer1977263,Leer1977,Leer1979}
中提出的 MUSCL 格式是克服 Godunuv 定理对精度阶数限制的最早尝试之一.
Harten \cite{Harten1983} 1983年引入总变差不增(Total Variation
Diminaishing, TVD) 的概念. TVD 条件的提出是双曲守恒律计算重要的进
步, 成为之后双曲守恒律计算格式的设计的主要理论依据之一. 上文中提到
的通量限制器和斜率限制器是两种设计 TVD 格式的手段. 为了克服 TVD 方
法在极值点处的降阶问题, Shu \cite{Shu1987} 引入 TVB 条件. 此后
以 Harten 为代表的学者提出本质无震荡 (non-oscillatory) 的思想, 发
展了本质无震荡 \cite{Harten1987231} (Essentially Non-Oscillatory,
ENO) 格式, 通过自适应选择计算模板的方式将 Godunov 方法及二
阶 MUSCL 重构推广到任意高阶, 从另外一个角度克服了TVD 格式极值点处
降阶的问题. 在其基础上 Shu, Osher 等学者进一步发展了加权本质无震
荡\cite{Liu1994200,Jiang1996}(Weighted Essentially
Non-Oscillatory, WENO)格式. WENO 格式是目前高阶格式的主流方法之
一 \cite{Wang2007}.

% =================================================================

上文提到的绝大多数计算格式都是针对一维情形. 相比高维情形, 双曲守恒
律在一维情形的理论相对完善, 在一定程度上可以为算法设计提供依据. 而
实际应用中的计算问题大多是二维三维问题. 因此双曲守恒律计算的一个挑
战就是将如何相对成熟的一维计算方法推广到高维, 尤其是推广到高维非结
构网格情形 \cite{Deng2012}.

随着 CFD 计算技术的发展, 非结构网格技术正逐渐受到更多的关注. 实际
应用问题, 例如整机模型的模拟、发动机模拟等都涉及到复杂的边界外
形. 生成复杂区域的结构网格是极其困难的. 代表性方法如嵌套网
格 (overset grid, chimera grid) 和多块网格(multi-block) 技术需要大
量的人工操作\cite{Venkatakrishnan1996,overset1983,Vatsa1993}. 而非
结构网格生成技术, 如常用的 Deluany 方法和阵面推进法 (Advancing
Front Method or Moving Front
Method)\cite{Anderson199423,Blazek2001,Ldhner1988}, 则可以方便的
实现自动网格生成, 从而在一定程度上降低网格生成的时间开销. 另一方
面, 非结构网格也具有结构网格所不具备的灵活性. 在非结构网格上可以更
方便的应用自适应算法. 在网格生成时也可以灵活的调整局部的稀疏性, 例
如通常希望在曲率较大的贴体表面增加网格的密度.

有效的单调性条件在非结构网格情形尤其重要. Spekreijse
\cite{spekreijse1987} 引入最早的高维单调性概念, 并以此为依据设计了
一系列保单调格式. Barth \cite{Barth1989} Liu
\cite{Liu1993} 和 Batten \cite{Batten1996} 等学者在此基础上发展出
列针对高维非结构网格双曲守恒律斜率限制策略.  Jameson
\cite{Jameson1995} 提出的 Local Extremum Diminishing(LED) 准则是对
一维 TVD 准则在高维情形的一个推广.  Hubbard 1999
\cite{Hubbard1999}注意到高维格式的设计大多要遵循某种形式的局部极大
值准则 (Local Maximum Principle), 提出斜率的极大值准则区域 (MP
region)的概念. 为相关的算法设计提供了统一的框架. 也使得斜率限制器
的设计可以绕开单自由度的斜率限制子的限制, 从而使限制过程可以在更大
程度上体现流体的高维性质. Hoteit \cite{Hoteit2004}、Buffard
\cite{Buffard2010}提出在 MP region 上求解一个约束优化问题从而得到
最合理的斜率估计. 但是求解优化问题的过程较为复杂.  Park、Kim
\cite{Park2010,Park2011,Park2014}通过扩大 MP 条件模板的方式引
入 MLP (Multi-dimensional Limiting Process) 条件, 实现对多维性质更
精细的刻画.

文献 \cite{Wang2007,Wang2013} 总结了最近几年计算流体力学高精度方法
方面的研究进展. 尽管高精度方法在模拟复杂流场时可以刻画更精细的解结
构,但是在包括方法的健壮性、内存占用、限制器设计等方面仍然处于不完
善的状态. 尤其是在非结构网格情形, 算法的设计十分困难. 所以目前工业
界仍然以相对成熟的二阶 MUSCL 类型的格式为主流计算方法. 近年来有一
系列对 MUSCL 类型方法的改进工作. Shu、Cockburn
\cite{Cockburn1998} 引入 MUSCL 类型 TVB 斜率限制器.
Clain、Buffard \cite{Clain2010,Buffard2010} 等学者提出了多斜率
的 (multislope) MUSCL 方法, 在每一个单元界面方向上引入一个单独斜
率, 这使得斜率的求解可以退化到一维的情形. 但是网格需要满足一定的正
则性以确保格式的稳定性. Hou 2014 \cite{Hou2015} 将多斜率的 MUSCL方
法成功应用于求解二维浅水方程. Touze 2015
\cite{LeTouze2015} 则将 该方法推广到一般的非结构网格, 并且消除了对
网格正则性的要求. 上文提到的 MLP 限制策略是对 MUSCL 格式改进的另外
一个重要尝试. 该格式在原始单斜率的 MUSCL 格式的基础上, 通过扩大限
制器模板的方法减少数值耗散, 以达到更高的计算精度. 但是限制过程需要
涉及到所有与本单元共享节点的单元.  因此这种模板相比只用 von
Neumann 邻居单元作为限制器模板要宽很多, 这一问题在三维的情形尤其突
出. 为避免在限制本单元时访问过多的邻居单元, 本文设计了一种新的计算
策略, 避免了对邻居单元的直接访问.

\section{间断有限元方法}
\label{sec:dg-method}

间断有限元方法 (Discontinuous Galerkin Method, DGM) 最早是 1973 年
由 Reed、Hill \cite{Reed1973} 提出, 并应用于中子输运方程的求解.
Shu、Cockburn \cite{Cockburn1989,Cockburn1990,Cockburn1998} 等学者
将其推广到求解双曲守恒律问题. 经过最近二十多年的发展, 间断有限元方
法已逐渐应用到许多应用领域中, 如半导体器件模拟、航空航天、多孔介质
问题、海洋模拟等 \cite{liuruxun2003}.  该方法结合了有限元和有限体
积法各自的优点.  类似有限元方法, 使用高阶多项式逼近而不是更宽的模
板来达到高阶精度.  另一个方面, 单元边界处因为解的间断性而需要求解
的 Riemann 问题, 可以采用有限体法中相同的处理方式. 事实上一阶的间断
有限元 $DG(0)$ 就等同于传统的 cell-centered 有限体积法.  从这个意
义上来说, 高阶的间断有限元 $DG(p), p>0$ 可以看做是有限体积法的自然
推广. 间断有限元方法主要有下面一些特点
\begin{itemize}
\item 作为有限元方法的一种特殊情况, 间断有限元方法可以十分自然的应
  用到非结构网格上, 从而可以处理复杂的计算区域. 甚至可以适用于更一
  般的网格, 例如任意三角形和四面体剖分或者带有``悬点''的非协调网格.
\item 每个计算单元相对独立, 计算过程具有高度紧致性. 这使得间断有限
  元方法可以达到极高的并行效率\cite{Juan2009}. 根据相关文献的计算
  结果, 其在非自适应和自适应两种情况下分别能够达到90\% 和 80\% 的
  并行效率 \cite{Remacle2003,Biswas1994}.
\item 基函数构造的灵活性使得 hp-自适应策略可以方便的应用于间断有限
  元方法.
\item 有良好的数学理论作为支撑. 间断有限元方法有良好的稳定性, 甚
  至可以证明其满足某些形式下的熵条件\cite{Jiang1994,Hou2006}.
\end{itemize}

同其他高阶方法, 间断有限元方法同样面临在解的间断处产生数值震荡的问
题. 需要在每一个时间步后应用一定的限制策略来抑制震荡. 有限体积法中
的部分限制器可以直接应用到间断有限元方法, 例如上文提到的基于斜率限
制器的限制策略. 这种限制策略可以自然地适用于一阶间断有限
元 $DG(1)$. 进行限制时为了达到某种稳定性条件, 只需要修改代表斜率的
高阶自由度. 但是对于高阶间断有限元 $DG(p), p>1$ 除代表斜率的一阶自
由度之外, 还有更高阶的自由度.  一个常用的做法是引入了一个判断是否
需要进行限制的判别准则, 所谓坏单元指示子 (Trouble Cell
Indicator). 仅在强间断附近等需要进行限制的单元上应用限制策略, 光滑
区域仍保持原自由度不变.  文献 \cite{Qiu2005b} 对几种指示子进行了总
结和对比. 关于坏单元的处理, Shu、Cockburn \cite{Cockburn1998} 将需
要进行限制的单元解投影到 $P_{1}$ 空间, 并应用斜率限制器. 但是
这样做将严重降低间断附近的精度.  同时, 坏单元指示子通常难以区分极
值点和强间断, 而在极值点处应用低阶限制器会破坏整体解的高精
度 \cite{Luo2007}. 为克服这一点, 近几年,有一系列文章讨论
将 WENO  与间断有限元方法相结
合 \cite{Zhu2008,Zhu2013,Qiu2005a}, 得到了精度更好的计算结果.  在
这类方法中, 如果判断某个网格上需要对数值解进行限制时, 则利用其周围
几层网格上解的自由度重构一个与 DG 方法同阶的 WENO多项式取代原
来的解. 但是引入 WENO 的计算模板也在一定程度上破坏了 DG 方法的紧致
性.

在这篇文章里, 我们只讨论有限体积法 MUSCL 类型斜率限制器在间断有限
元方法上的应用, 对 MLP 类型 MUSCL 斜率限制器的实现策略做了一些改
进, 取得了很好的计算效果.

\section{并行计算}
\label{sec:parallel}

近半世纪以来, 微处理器的性能提升主要依赖于制程尺寸的不断缩小, 即通
过增加单位面积上的晶体管数量来提高计算速度. 但是随着制程尺寸逐渐接
近硅材料的物理极限, 新工艺的研发难度剧增. 另一方面, 芯片的散热问题
也正逐渐接近空气制冷的极限 \cite{Pacheco2011}. 因此单纯的通过增加
微处理的性能来提高计算速度已经变得非常困难, 延续半个世纪之久的摩尔
定律正在逐渐失效 \cite{Moore1998,Waldrop2016}. 但是在现代航空航天、
医疗、金融、科研等众多领域中, 对计算性能的需求仍是一个重要的瓶
颈. 尽管单个芯片的性能提升正在放缓, 我们可以通过并行计算的方式来继
续提升计算性能. 并行计算指的是基于某种通讯手段协同多个微处理器协同
处理某个计算问题, 从而提高计算性能. 这也几乎成为目前解决大规模科学
计算问题的唯一有效途径.

并行计算在计算流体力学中有着广泛的应用. 其处理方式通常是将计算区域
按处理器个数分解为若干子网格. 每个进程负责处理其中的一块计算区域,
计算过程中通过处理器间的数据通信实现整体的协同.

并行效率是并行计算的一个重要的衡量指标. 并行效率一方面受硬件环境的
影响, 例如通信的带宽以及芯片之间的性能差异. 另一方面并行效率在很大
程度上取决于计算方法. 如何提高并行效率是相关算法研究的一个重要课
题. 如果算法本身需要在进程之间进行大量的数据交换, 那么通讯时间可能
会占用计算时间的大部分. 因此, 子问题之间的关联程度越小的计算方法越
适合做并行化. 间断有限元方法在计算过程只需要边相邻单元(von
Neumann 邻居单元)的信息, 而有传统方法如限体积法则需要多层邻居单元
的信息以实现高阶精度. 这使得间断有限元方法在并行效率方面有着巨大的
优势.

MPI (Massage Passing Interface)、OpenMP (Open
Multi-Processing) 和 CUDA (Compute Unified Device Architecture)是
当前主流的并行计算实现框架. 本文采用 MPI 分布式并行模式.

\section{本文的主要内容}
\label{sec:main-containt}

本文的主要内容是关于高维非结构网格解法器的实现以及 MUSCL 类型斜率限
制器的改进. 主要包含下面几个方面,
\begin{itemize}
\item 在 MLP 斜率的限制器的框架下, 提出了一种新的实现策略. 该策略
  利用节点作为相邻单元之间信息传递的媒介. 限制器作用过程中首先将数
  值信息归集到依托于节点的数据结构, 在计算斜率限制器时, 只需要访问
  节点处的数据, 因而避免了对模板内邻居单元的直接访问.
\item 开发了二维非结构网格间断有限元解法器. 实现了不同的 MUSCL 类
  型斜率限制器. 将 MLP 限制器应用于间断有限元方法, 并实现了改进策
  略. 通过数值实验, 将 MLP 限制器与其他限制策略进行了对比.
\item 完善了基于 PHG 平台 \cite{Zhang2009} 的三维非结构网格并行自
  适应有限体积求解器 Libfvphg. 添加数值算例, 测试并行效率等.
\end{itemize}

本文的余下部分结构安排如下. 第二章介绍双曲守恒律有限体积法的空间离
散、数值通量计算、限制器、非结构网格生成等技术. 并简单介绍几种非结
构网格 MUSCL 类型斜率限制器方法及相关的理论. 第三章讨论间断有限元
方法以及 MLP 类型限制器在间断有限元方法中的应用, 以及基于二维非结
构网格求解器的数值实验. 第四部分是关于对基于 PHG 的平台的有限体积
法求解器Libfvphg 的改进. 并添加经典测试算例, 进行并行效率测试. 最
后一章对文章内容进行进行总结和展望.
